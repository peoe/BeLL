\section{Druckvorgang}
%\todoinline{Beschreibung, warum die Bauteile auf- und verteilt werden müssen}
%genaue Maße der Druckplatte; siehe: 
%\verb|https://eu.makerbot.com/fileadmin/Inhalte/Support/Manuals/German_UserManual_V.4_Replicator2.pdf|
%oder Kapitel 4 unten
% [28.5 x 15.3 x 15.5 cm]
Nachdem die einzelnen Bestandteile des Modells berechnet wurden, können diese gedruckt werden.
Hierbei ist allerdings zu bedenken, dass alle einzelnen Bauteile nicht in einem Druckvorgang gedruckt werden können, da die Grundplatte des 3D-Druckers zu klein ist. \\
In unserem Fall wies diese eine Länge von 28,5 cm und eine Breite von 15,2 cm auf (siehe Quelle \cite{makerbotspecs}).
\subsection{Minimierung der Druckvorgänge}
Um die Dimensionen der Grundplatte effektiv auszunutzen, werden einzelne Elemente des Modells nach der Größe sortiert und dann solange nebeneinander angeordnet, wie es die Breite der Druckfläche zulässt.
Anschließend wird eine neue Reihe eröffnet, in der nun neue Elemente platziert werden. 
Dieser Prozess wird solange fortgeführt, bis die Länge der Druckfläche überschritten werden würde.
Dann wird eine neue Gruppierung von Objekten bzw. \icode{Union} erstellt, in welcher der Algorithmus weiterläuft.
Das Resultat zeigt eine Liste von Objekten, die eine platzeffiziente Anordnung dieser repräsentiert.
Jedes einzelne Element in der Liste kann mit einem Durchgang gedruckt werden.
\subsection{Ermittlung der Größen}
Das Einzige, was die Strukturen \icode{Wall}, \icode{Corner} und \icode{BasePlate} in diesem Algorithmus voneinander unterscheidet, ist die Größe, welche unterschiedlich bestimmt wird.
Da die Anordnung ein zweidimensionales Problem ist, wird die Höhe der Objekte nicht berücksichtigt.
\subsubsection{Wände}
Die Wandelemente werden durch eine \icode{Edge} dargestellt und so modifiziert, dass sie in die \icode{Corner} Elemente gesteckt werden kann.
Sie sind dadurch immer kleiner als die \icode{Edge}, deswegen wird dessen Länge als Länge der Wand gesetzt. 
Die Breite ermittelt sich durch die definierte \icode{wallWidth} der übergebenen \icode{Params}--Klasse.
\subsubsection{Eckpfeiler}
Für die Größenberechnung der Eckpfeiler wird der längste Pin, welcher an dem jeweiligen Objekt anliegt festgestellt.
Die Länge dieses Pins ist dann die Seitenlänge eines Quadrats, welche die Ausmaße des Eckpfeilers angibt.
\subsubsection{Grundplatten}
Um die optimale Größe der Grundplatten zu berechnen, muss das minimale Begrenzungsrechteck, die OMBB, dieser berechnet werden.
Dafür wird die OMBB des \icode{Face} ermittelt, welches die Grundplatte darstellt und anschließend vergrößert, damit Modifikationen, die an der dieser durchgeführt wird, berücksichtigt werden.
\paragraph{Bestimmung des Convex Hulls}
Die Berechnung der OMBB wird über diesen Zwischenschritt realisiert, da nach dem von Freeman und Shapira bewiesenen Satz, eine Seite des minimalen Begrenzungsrechtecks kollinear mit einer der konvexen Hülle sein muss. \\
Für die Ermittlung wird der \q{Gift Wrapping Algorithm} herangezogen.
Bei diesem wird zuerst der Punkt mit der kleinsten Ordinate als Startpunkt $P_0$ festegelegt.
Gibt es mehrere von diesen, wird aus denen derjenige Punkt mit der kleinsten Abszisse gewählt.
Von $P_0$ ausgehend wird eine Gerade durch einen beliebigen Punkt $P$ des Polygons gebildet und anschließend überprüft, ob es einen Punkt $S$ gibt der links von dieser liegt.
Im Programm geschieht das durch die Berechnung des Winkels zwischen den Vektoren $\overrightarrow{{P_0}P}$ und $\overrightarrow{{P_0}S}$ mithilfe der \icode{Vector.angleTo(Vector v)} Methode.
Ist dieser Winkel kleiner als $180^\circ$, liegt $S$ links von der Gerade durch $P_0$ und $P$.
$S$ wird folglich als neuer Punkt $P$ gesetzt und die Berechnung fortgesetzt.
Dies wird solange durchgeführt bis alle Punkte überprüft wurden und ein nächster Punkt $P_1$ der konvexen Hülle feststeht.
Der Algorithmus wird nun fortgeführt mit $P_1$ als Ausgangspunkt, solange bis der Anfangspunkt $P_0$ wieder erreicht wurde, so dass die konvexe Hülle \q{geschlossen} ist.
\paragraph{Bestimmung der OMBB}
Dafür wird die OMBB(Oriented Minimum Bounding Box) des Polygons berechnet und davon ausgehend ein Winkel, um den die Grundplatte rotiert werden muss, damit diese Kanten des Rechtecks parallel zu den Seiten der Druckfläche sind.
Zuerst der Convex Hull, die \q{konvexe Hülle} ermittelt.