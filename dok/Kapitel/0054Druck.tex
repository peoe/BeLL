\section{Druckvorgang}
%\todoinline{Beschreibung, warum die Bauteile auf- und verteilt werden müssen}
%genaue Maße der Druckplatte; siehe: 
%\verb|https://eu.makerbot.com/fileadmin/Inhalte/Support/Manuals/German_UserManual_V.4_Replicator2.pdf|
%oder Kapitel 4 unten
% [28.5 x 15.3 x 15.5 cm]
Nachdem die einzelnen Bestandteile des Modells berechnet wurden, können diese gedruckt werden.
Hierbei ist allerdings zu bedenken, dass alle einzelnen Bauteile nicht in einem Druckvorgang gedruckt werden können, da die Grundplatte des 3D-Druckers zu klein ist. \\
In unserem Fall wies diese eine Länge von 28,5 cm und eine Breite von 15,2 cm auf (siehe Quelle \cite{makerbotspecs}).
Um die Dimensionen der Grundplatte effektiv auszunutzen, werden einzelne Elemente des Modells zu Objektgruppen zusammengefasst, die nebeneinander platzeffizient gedruckt werden können.

\subsection{Aufteilen der Grundplatten}
\todoinline{Beschreiben, wie das gemacht wird}

\subsection{Verteilung der Bauteile}
\todoinline{Beschreibung der Verteilung einzelner Bauteile für passenden Druckvorgang}
\subsubsection{Wände}
Die Wandelemente
\subsubsection{Eckpfeiler}
Für die Ausgabe der druckfähigen, platzoptimierten Eckpfeiler werden diese zuerst nach ihrer Größe sortiert.
Die Größe eines Eckstückes ist dabei die Länge des längsten anliegenden Pins.
Aus dieser kann ein Quadrat gebildet werden, welches das komplette auf 2D projizierte 3D-Objekt beinhaltet.
\todo{Eventuell Größe bei Eckpfeiler erwähnen}
Anschließend wird, beginnend mit dem kleinsten Element, alle Eckpfeiler in Reihen angeordnet. 
Wäre dabei die aktuelle Reihe länger als die maximale Druckbreite, wird eine neue Reihe begonnen, die direkt neben der vorherigen liegt.
Ist durch die Erstellung einer neuen Reihe die maximale Drucklänge überschritten, beginnt eine neue \icode{Union} die mit Eckpfeilern gefüllt wird.
Alle entstandenen Unions bzw. Gruppierungen von Eckpfeilern können nun mit einem Durchgang gedruckt werden  
\subsubsection{Grundplatten}
