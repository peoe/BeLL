\chapter{Vorgehen zur Problemlösung}
Die Anwendung konvertiert eine .dxf Grundrissdatei in .scad und .stl Dateien, die zum Drucken benutzt werden können.
Dafür sind mehrere Zwischenschritte nötig.
\section{Einlesen des Grundrisses}
Den Beginn der Verarbeitung markiert die Grundrissdatei, in welcher sämtliche Werte, die im weiteren Verlauf des Programmes relevant werden, enthalten sind.

\subsection{Funktionsweise der Benutzeroberfläche}
Die Benutzeroberfläche (GUI) setzt sich zusammen aus einem \icode{JFrame}, in dem zwei \icode{JTextField}, ein \icode{FileChooserButton}, ein \icode{StartButton} und ein \icode{ShowResultButton} platziert sind.
Der Ausgangszustand der Benutzeroberfläche (GUI) ist in Abb. \thebildnrnext\ zu sehen.
Der \icode{StartButton}, der \icode{ShowResultButton} und der \icode{FileChooserbutton} erben jeweils von der \icode{JButton}-Klasse. \\

\begin{Bild}{Ausgangszustand der Benutzeroberfläche (Screenshot der Verfasser)}
	\includegraphics[width = 100mm]{Bilder/GUI/GUI_Startup}
\end{Bild}

Der Nutzer kann im Ausgangszustand über den \icode{FileChooserButton} einen \icode{JFileChooser}-Dialog öffnen, mit dem er die Datei, die er umwandeln möchte, auswählen kann.
Sobald er dann eine Datei ausgewählt hat, wird der Dateipfad zu dieser Datei zusätzlich im \icode{JTextField} angezeigt.
Alternativ zum \icode{JFileChooser}-Dialog kann der Nutzer auch direkt in das links positionierte \icode{JTextField} den Dateipfad eingeben.
Im \icode{JFileChooser} ist außerdem ein \icode{FileFilter} implementiert, der dem Nutzer lediglich .dxf-Dateien anzeigt.
Der \icode{FileFilter} verhindert dabei, dass der Nutzer einen anderen Dateitypen auswählt.
Im \icode{JTextField} darunter gibt der Nutzer den Ordnernamen ein, in dem die umgewandelten Dateien ausgegeben werden. \\
Der \icode{StartButton} überprüft nach der Interaktion mit dem \icode{JTextField}, ob eine Datei ausgewählt und ein Ordnername eingegeben wurde.
Sollten beide dieser Bedingungen erfüllt sein, wird der \icode{StartButton} aktiviert und der Nutzer kann die Konvertierung starten (siehe Abb. \thebildnrnext).\\

\begin{Bild}{Benutzeroberfläche mit aktiviertem \icode{StartButton} (Screenshot der Verfasser)}
	\includegraphics[width = 100mm]{Bilder/GUI/GUI_Convert_Ready}
\end{Bild}

Beim Klicken des \icode{StartButton} wird überprüft, ob das Programm die ausführbare Datei von OpenSCAD unter \icode{C:\textbackslash \textbackslash Program Files\textbackslash OpenSCAD\textbackslash } finden kann.
Diese Datei wird benötigt, um die Umwandlung der .scad-Dateien in .stl-Dateien zu ermöglichen.
Sollte diese dort nicht gefunden werden, wird der Nutzer mit einem Dialog darauf hingewiesen (siehe Abb. \thebildnrnext). \\

\begin{Bild}{Dialog zur Warnung des Nutzers (Screenshot der Verfasser)}
	\includegraphics[width = \textwidth]{Bilder/GUI/GUI_SCAD_Error}
\end{Bild}

Nach dem Schließen des Dialogs öffnet sich nun ein weiterer \icode{JFileChooser} (siehe Abb. \thebildnrnext).
Mit diesem hat der Nutzer die Möglichkeit, den Pfad zur ausführbaren Datei anzugeben. \\

\begin{Bild}{Dialog zum Verlinken der openscad.exe (Screenshot der Verfasser)}
	\includegraphics[width = 120mm]{Bilder/GUI/GUI_SCAD_Linking}
\end{Bild}

Sobald dieser Vorgang beendet ist, bleiben dem Nutzer nun die Möglichkeiten, den Ordner mit den Ausgabedateien anzuzeigen oder einen anderen Grundriss umzuwandeln.\\

\subsection{Funktionsweise der Bibliothek \textit{kabeja}}
Das Einlesen der Daten eines Grundrisses erfolgt mit der Java-Bibliothek „kabeja“. 
Diese ermöglicht es, aus .dxf-Dateien alle DXF-Objekte eines bestimmten Typs zu erhalten, deren Werte in einer Liste zu speichern und später zu verarbeiten. \\

\begin{code}[DXF File Parser]
	public static ArrayList<Line> getAutocadFile(String filePath) throws ParseException {
		ArrayList<Line> vcs = new ArrayList<>();
		/*Ermitteln des DXFDocuments doc*/
		List lst = doc.getDXFLayer("0").getDXFEntities(/*...*/);
		for (int index = 0; index < lst.size(); index++) {
			DXFLine dxfline = (DXFLine) lst.get(index);
			Line v = new Line(<Vektor des Anfangspunktes>, <Vektor des Endpunktes>);
			vcs.add(v);
		}
		return vcs;
	}
\end{code}

Aus der .dxf-Datei deren Pfad der Klasse \q{DXFReader} übergeben wird, werden alle DXF-Objekte, die mit dem Typen \icode{DXFLine} übereinstimmen, in einer Liste zurückgegeben (vgl. Codeausschnitt~1). 
Die Koordinaten der Start- und Endpunkte der DXFLines  in dieser Liste werden anschließend in eine Liste von Linien übertragen, welche unter anderem bei der Umwandlung des Graphen in die DCEL Verwendung findet.