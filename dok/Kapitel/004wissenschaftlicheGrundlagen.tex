\chapter{Grundlagen}
\section{OpenSCAD}
\section{planarer Graph}
Ein planarer, auch plättbarer Graph ist ein Graph der in der Ebene mithilfe von Punkten bzw. Knoten (1) und Kanten (2) dargestellt werden kann, ohne dass sich die Kanten schneiden. 
Dabei findet eine Einteilung in Gebiete bzw. Flächen durch die Kanten statt. 
Sie bilden den Rand einer Fläche (3). 
Die Fläche um den Graphen herum wird äußerstes Gebiet genannt.
\section{Doubly connected edge list}
Um planare Graphen ohne Informationsverlust zu speichern werden in der Informatik Referenzen zwischen den einzelnen Bestandteilen des Graphen eingesetzt. 
In der sogenannten \q{Doubly connected edge list} (DCEL) erhält eine Kante, die aus einem Anfangsknoten und Endknoten besteht jeweils eine Vorgänger-, eine Nachfolger- und eine Zwillingskante. 
Jedem Knoten wird eine ausgehende Kante und den Flächen eine anliegende Kante zugewiesen. Diese Verknüpfungen ermöglichen es, ausgehend von einem Element direkt auf andere zu schließen, indem  …
OpenScad
\todo{Openscad dazu, noch was schreiben}